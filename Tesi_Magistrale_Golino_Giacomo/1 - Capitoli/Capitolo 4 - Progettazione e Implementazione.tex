\chapter{Progettazione e implementazione dell'interfaccia utente}
\label{chap:design-implementation}

\section{Introduzione al capitolo}
Questo capitolo approfondisce la progettazione dell'interfaccia utente del sistema sviluppato, con l'obiettivo di mostrare come l'integrazione delle funzionalit\`a di Trigger-Action Programming (TAP) nel Gemello Digitale sia stata tradotta in un'esperienza di interazione coerente, controllabile e aderente ai requisiti discussi nei capitoli precedenti. L'interfaccia non \`e un semplice elemento di presentazione, ma il punto di accesso privilegiato per osservare lo stato della Smart Home e per definire automazioni che, dopo la fase di simulazione, vengono applicate alla componente fisica.

In continuit\`a con l'architettura descritta nel Capitolo~\ref{chap:digital-twin-smart-home}, l'interfaccia svolge una funzione di mediazione tra l'utente e i moduli del sistema: rende percepibili i dati generati dal Gemello Digitale, permette di interpretare i risultati della simulazione e guida l'utente nella creazione e gestione delle regole TAP. Il capitolo si colloca quindi come ponte tra la visione concettuale e l'implementazione, motivando le scelte di design e illustrando il percorso dal prototipo all'interfaccia funzionante.

\section{Analisi dei requisiti dell'interfaccia}
L'analisi dei requisiti prende avvio dal contesto applicativo della Smart Home e dal paradigma TAP presentato nel Capitolo 2, nonch\'e dalla struttura modulare del Gemello Digitale descritta nel Capitolo~\ref{chap:digital-twin-smart-home}. In questo quadro, l'interfaccia deve rendere accessibile un insieme eterogeneo di funzionalit\`a, senza introdurre complessit\`a non necessarie o ambiguit\`a nella gestione delle automazioni.

Dal punto di vista funzionale, l'interfaccia deve consentire la consultazione dello stato del sistema, la visualizzazione dei consumi energetici e delle previsioni, la creazione e modifica di regole trigger-action, la gestione delle configurazioni dell'abitazione e delle preferenze utente, nonch\'e l'esecuzione di controlli manuali quando necessario. \`E inoltre essenziale fornire un accesso esplicito agli esiti delle simulazioni, includendo eventuali conflitti o suggerimenti, poich\'e la validazione preventiva costituisce un requisito cardine del sistema.

I requisiti di usabilit\`a e chiarezza discendono dalla natura del dominio: le automazioni influenzano il comportamento della casa reale e un errore di configurazione pu\`o produrre conseguenze tangibili. Per questo motivo l'interfaccia deve privilegiare la visibilit\`a dello stato, il feedback tempestivo sulle azioni, la tracciabilit\`a delle informazioni e l'adozione di una terminologia coerente con quella introdotta nei capitoli precedenti. La prevenzione degli errori non si limita a messaggi di avviso, ma richiede una progettazione che renda evidenti le relazioni tra trigger, condizioni e azioni, riducendo l'ambiguit\`a nelle scelte dell'utente.

La tipologia di utenti prevista comprende soggetti con livelli di competenza differenti: utenti domestici interessati a monitorare lo stato della casa e a configurare automazioni semplici, utenti tecnici chiamati a definire mappe e dispositivi, e profili orientati all'analisi dei consumi che richiedono viste sintetiche ma confrontabili nel tempo. La progettazione ha quindi dovuto conciliare immediatezza e profondit\`a informativa, prevedendo percorsi che consentano sia un utilizzo rapido, sia un accesso graduato ai dettagli.

\section{Progettazione dell'interfaccia utente}
\subsection{Approccio progettuale e criteri guida}
La progettazione ha seguito un approccio user-centered, con iterazioni mirate alla definizione dei principali flussi di interazione: monitoraggio dello stato della casa, consultazione dei consumi, definizione delle regole TAP e gestione della configurazione. La coerenza con il Gemello Digitale ha richiesto che ogni entit\`a significativa nel sistema, come ambienti, dispositivi e automazioni, fosse riconoscibile nell'interfaccia con una terminologia stabile e una rappresentazione uniforme.

Tra i criteri guida si \`e privilegiata la semplicit\`a operativa, intesa come riduzione del carico cognitivo e come minimizzazione dei passi necessari per completare un compito. Parallelamente, la modularit\`a dell'interfaccia \`e stata considerata essenziale per garantire l'estendibilit\`a del sistema e per permettere una separazione netta tra consultazione e azione. Tale separazione, oltre a favorire l'usabilit\`a, risponde alla necessit\`a di prevenire attivazioni non intenzionali in un contesto in cui le automazioni incidono sul mondo fisico.

\subsection{Layout, navigazione e organizzazione delle informazioni}
L'architettura informativa \`e stata organizzata in sezioni principali, accessibili tramite una navigazione persistente. La dashboard assolve il ruolo di sintesi dello stato del Gemello Digitale, mentre le sezioni dedicate a consumi, automazioni e configurazione consentono di approfondire aspetti specifici senza perdere il contesto generale. Questa scelta risponde alla necessit\`a di mantenere il controllo percettivo del sistema anche quando si accede a funzionalit\`a avanzate.

L'organizzazione interna delle schermate segue una logica di progressiva rivelazione: gli indicatori essenziali sono presentati in modo immediato, mentre i dettagli vengono esposti solo quando l'utente decide di approfondire. Tale scelta \`e motivata dall'eterogeneit\`a dei dati trattati dal Gemello Digitale e dalla necessit\`a di evitare un sovraccarico informativo, specialmente per gli utenti meno esperti. La disposizione dei controlli di azione in aree dedicate, distinte dalle aree di sola consultazione, rafforza inoltre la chiarezza operativa e la prevenzione degli errori.

\subsection{Rappresentazione delle regole Trigger-Action e feedback}
L'integrazione del paradigma TAP ha richiesto una rappresentazione che rendesse esplicita la sequenza logica di trigger, condizioni e azioni. L'interfaccia \`e stata quindi progettata per guidare l'utente attraverso la composizione delle regole, assicurando che ogni passaggio sia comprensibile e verificabile. La scelta di una struttura modulare risponde alla necessit\`a di mantenere la flessibilit\`a del paradigma TAP senza introdurre ambiguit\`a nei legami tra gli elementi della regola.

Un aspetto centrale riguarda la restituzione dei risultati della simulazione. Poich\'e la validazione preventiva \`e parte integrante dell'architettura, l'interfaccia deve comunicare in modo chiaro gli esiti, includendo eventuali conflitti e suggerimenti. Ci\`o richiede un sistema di feedback che distingua tra avvisi informativi e blocchi operativi, conservando la trasparenza dell'intero processo. In tal modo l'utente non percepisce la simulazione come un vincolo opaco, ma come un supporto decisionale coerente con gli obiettivi del Gemello Digitale.

\section{Prototipazione tramite FIGMA}
La prototipazione ad alta fedelt\`a \`e stata realizzata in FIGMA per validare la struttura informativa e il flusso di interazione prima dell'implementazione. Le immagini del prototipo sono disponibili nella cartella \texttt{Tesi\_Magistrale\_Golino\_Giacomo\textbackslash images\textbackslash Chap - 4} e vengono richiamate nelle figure di questa sezione per illustrare le principali schermate.

\subsection{Schermate principali}
La sezione dedicata ai consumi energetici, mostrata in Figura~\ref{fig:cap4_consumptions}, rappresenta il punto di accesso alle analisi del consumo e alle previsioni. La schermata \`e concepita per supportare il confronto tra intervalli temporali differenti e per distinguere tra consumo totale, contributo delle automazioni e stime predittive. Nel prototipo sono presenti varianti della vista che mostrano tali configurazioni, coerenti con i requisiti di supporto alle decisioni energetiche.

\begin{figure}[H]
    \centering
    \includegraphics[width=1\linewidth]{images/Chap - 4/Consumptions.png}
    \caption{Prototipo della sezione di analisi dei consumi energetici.}
    \label{fig:cap4_consumptions}
\end{figure}

La gestione delle automazioni (Figura~\ref{fig:cap4_automations}) esprime la componente centrale dell'integrazione TAP nel Gemello Digitale. La schermata \`e pensata per rendere visibili le regole attive, lo stato di ciascuna automazione e l'accesso ai dettagli necessari per la modifica. In questo contesto, la relazione tra trigger, condizioni e azioni \`e mantenuta esplicita per facilitare il controllo e la comprensione delle regole.

\begin{figure}[H]
    \centering
    \includegraphics[width=1\linewidth]{images/Chap - 4/Automations.png}
    \caption{Prototipo della sezione di gestione delle automazioni trigger-action.}
    \label{fig:cap4_automations}
\end{figure}

La configurazione dell'abitazione, illustrata in Figura~\ref{fig:cap4_house_configuration}, supporta la mappatura tra spazi fisici e rappresentazione digitale. Tale passaggio \`e essenziale per garantire che le automazioni TAP siano contestualizzate in modo coerente con la Smart Home reale, preservando la fedelt\`a del Gemello Digitale.

\begin{figure}[H]
    \centering
    \includegraphics[width=0.95\linewidth]{images/Chap - 4/House configration_final_1.png}
    \caption{Prototipo della configurazione dell'abitazione e dei dispositivi.}
    \label{fig:cap4_house_configuration}
\end{figure}

L'area profilo (Figura~\ref{fig:cap4_profile}) integra la gestione delle informazioni personali e delle preferenze dell'utente, consentendo di collegare le impostazioni individuali ai parametri utilizzati dai moduli di simulazione e analisi. Questa schermata \`e fondamentale per mantenere coerenza tra obiettivi dell'utente e comportamento del sistema.

\begin{figure}[H]
    \centering
    \includegraphics[width=0.7\linewidth]{images/Chap - 4/Profile.png}
    \caption{Prototipo dell'area profilo e delle preferenze utente.}
    \label{fig:cap4_profile}
\end{figure}

La vista sull'impronta ecologica (Figura~\ref{fig:cap4_ecological}) fornisce una sintesi orientata alla sostenibilit\`a, utile per collegare il comportamento delle automazioni agli obiettivi energetici del sistema. La presenza di questa schermata nel prototipo rafforza la connessione tra TAP e ottimizzazione dei consumi.

\begin{figure}[H]
    \centering
    \includegraphics[width=0.9\linewidth]{images/Chap - 4/Ecological footprint.png}
    \caption{Prototipo della vista di sintesi dell'impronta ecologica.}
    \label{fig:cap4_ecological}
\end{figure}

\subsection{Elementi di supporto e coerenza visiva}
Oltre alle schermate principali, il prototipo include elementi di supporto che contribuiscono alla coerenza visiva e al feedback operativo. La definizione della palette cromatica (Figura~\ref{fig:cap4_palette}) esplicita i ruoli dei colori per stati, avvisi e informazioni neutre, garantendo una leggibilit\`a uniforme tra le sezioni. L'uso di tali convenzioni \`e funzionale alla chiarezza e alla riconoscibilit\`a delle informazioni.

\begin{figure}[H]
    \centering
    \includegraphics[width=0.7\linewidth]{images/Chap - 4/Color palette.png}
    \caption{Palette cromatica utilizzata nel prototipo.}
    \label{fig:cap4_palette}
\end{figure}

Il prototipo comprende inoltre esempi di notifiche contestuali per comunicare l'esito di azioni o variazioni di stato, come mostrato in Figura~\ref{fig:cap4_feedback}. Questi elementi sono rilevanti per la percezione di affidabilit\`a del sistema, in quanto informano l'utente sugli effetti delle operazioni eseguite.

\begin{figure}[H]
    \centering
    \includegraphics[width=0.7\linewidth]{images/Chap - 4/Achievements popup.png}
    \caption{Esempio di notifica contestuale nel prototipo.}
    \label{fig:cap4_feedback}
\end{figure}

\section{Scelte implementative}
La traduzione del prototipo in un'interfaccia funzionante ha richiesto decisioni che preservassero la coerenza concettuale e grafica, garantendo al contempo affidabilit\`a e manutenibilit\`a. L'adozione di un'architettura a componenti permette di riflettere la modularit\`a del design, assicurando uniformit\`a tra le sezioni e facilitando l'evoluzione futura dell'interfaccia. Tale impostazione supporta la riusabilit\`a di elementi comuni e riduce la probabilit\`a di inconsistenze visive o comportamentali.

L'interfaccia deve gestire dati dinamici provenienti da moduli diversi, con aggiornamenti asincroni e potenziali ritardi. Per evitare discrepanze percettive tra stato reale e rappresentazione digitale, sono stati previsti meccanismi di aggiornamento chiari e indicatori che rendono esplicito il momento di sincronizzazione. Questo aspetto \`e particolarmente rilevante per le automazioni TAP, poich\'e l'utente deve distinguere tra una regola definita e una regola effettivamente validata dalla simulazione.

Un ulteriore elemento riguarda la validazione delle regole e la prevenzione degli errori. Il flusso di definizione delle automazioni \`e stato implementato in modo guidato, con controlli che rendono evidenti i passaggi necessari e con feedback che segnalano incompletezze o conflitti prima dell'invio al modulo di simulazione. Tale scelta privilegia la sicurezza operativa rispetto alla mera rapidit\`a di configurazione, in linea con la natura critica delle automazioni in un ambiente domestico reale.

Nel passaggio dal prototipo all'implementazione sono stati affrontati compromessi legati alla densit\`a informativa. Alcuni dettagli avanzati sono stati mantenuti in sezioni dedicate, cos\`i da preservare la leggibilit\`a delle viste principali. Inoltre, la coesistenza con strumenti conversazionali esterni per la creazione delle automazioni, discussa nel Capitolo~\ref{chap:digital-twin-smart-home}, ha richiesto di mantenere un percorso coerente tra le diverse modalit\`a di interazione, evitando duplicazioni non necessarie.

\section{Considerazioni finali}
La progettazione dell'interfaccia utente ha seguito un percorso coerente con i requisiti e con l'architettura del Gemello Digitale, integrando il paradigma TAP in un'interazione comprensibile e verificabile. Le scelte effettuate hanno privilegiato la trasparenza del processo di simulazione, la chiarezza terminologica e la prevenzione degli errori, bilanciando la complessit\`a del dominio con l'esigenza di usabilit\`a.

Il capitolo conclusivo riprende queste scelte per discutere il sistema nel suo complesso e per sintetizzare le implicazioni progettuali e applicative emerse, fornendo un quadro finale delle opportunit\`a e delle prospettive di evoluzione.
