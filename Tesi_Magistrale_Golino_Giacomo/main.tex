\documentclass[a4paper,12pt,oneside]{book} % Formato documento: foglio a4, font size 12, solo fronte

% ================================================================================
% ENCODING & INTERNATIONALISATION
% ================================================================================
\usepackage[utf8]{inputenc}        % Consente di scrivere direttamente caratteri UTF-8
\usepackage[T1]{fontenc}           % Evita problemi con accenti/virgolette
\usepackage[italian]{babel}        % Imposta sillabazione, date e stringhe in italiano
\usepackage{csquotes}              % Virgolette “intelligenti” e compatibilità con biblatex

% ================================================================================
% PAGE LAYOUT & SPACING
% ================================================================================
\usepackage[
  lmargin=20mm,
  rmargin=20mm,
  tmargin=25mm,
  bmargin=20mm
]{geometry}                        % Margini della pagina
\linespread{1.5}                   % Interlinea 1.5
\setlength{\parindent}{0pt}        % Nessun rientro a inizio paragrafo
\setlength{\parskip}{1em}          % Spazio dopo i paragrafi
\setlength{\headheight}{15pt}      % Altezza dell’intestazione
\addtolength{\topmargin}{-2.5pt}   % Regola fine del margine superiore
\usepackage{fancyhdr} % Per modificare l'intestazione delle pagine

% ================================================================================
% LISTS & ENUMERATIONS
% ================================================================================

\usepackage{enumitem}              % Controllo fine su elenchi puntati/numerati
\usepackage{siunitx}               % Opzioni extra in enumitem (es. \setitemize)
\setitemize{noitemsep,topsep=0pt,parsep=0pt,partopsep=0pt} % Liste compatte
\setenumerate{noitemsep,topsep=0pt,parsep=0pt,partopsep=0pt}

% ================================================================================
% MATHEMATICS
% ================================================================================
\usepackage{amsmath}               % Ambiente matematico AMS
\usepackage{mathtools}             % Estensioni ad amsmath (es. \left\{ … \right. )

% ================================================================================
% GRAPHICS, FIGURES & FLOATS
% ================================================================================
\usepackage{graphicx}              % Inserimento immagini
\usepackage{float}                 % Ambiente [H] per il posizionamento forzato
\usepackage{pgfplots}              % Grafici vettoriali/TikZ
\pgfplotsset{compat=1.18}          % Compatibilità PGFPlots
\graphicspath{{images/}}           % Percorso di default per le immagini

% ================================================================================
% CAPTIONS & HYPERLINKS
% ================================================================================
\usepackage[table]{xcolor}         % Colori (anche per righe/colonne di tabelle)
\usepackage{caption}               % Personalizzazione didascalie
\usepackage[hidelinks]{hyperref}   % Collegamenti ipertestuali (senza cornici)
\usepackage{xurl}                  % Permette il line-break negli URL lunghi

% ================================================================================
% TABLES
% ================================================================================
\usepackage{array}                 % Colonne avanzate (m{…}, >{\centering} ecc.)
\usepackage{tabularx}              % Tabelle a larghezza variabile \textwidth
\usepackage{longtable}             % Tabelle multipagina
\usepackage{booktabs}              % Regole tipografiche \toprule \midrule \bottomrule
\usepackage{multirow}              % Celle verticalmente unite
\newcolumntype{C}[1]{>{\centering\arraybackslash}m{#1}} % Colonna centrata h/v
\makeatletter
\providecommand{\insert@pcolumn}{\insert@column} % Fix per compatibilità xcolor/array
\makeatother
\definecolor{lightgray}{HTML}{b0b7c2}                   % Grigio per righe alternate

% ================================================================================
% CODE LISTINGS
% ================================================================================
\usepackage{listings}              % Visualizzazione codice sorgente
% --- Lingua YAML personalizzata con stile ---
\lstdefinelanguage{yaml}{
  morecomment=[l]{\#},
  morestring=[b]",
  morestring=[b]',
  sensitive=true
}
\lstdefinestyle{yamlcolor}{
  language=yaml,
  backgroundcolor=\color{gray!10},
  basicstyle=\ttfamily\small,
  commentstyle=\color{gray},
  keywordstyle=\color{blue}\bfseries,
  stringstyle=\color{teal},
  showstringspaces=false,
  numbers=left,
  numbersep=5pt,
  numberstyle=\tiny\color{gray},
  frame=single,
  breaklines=true,
  captionpos=b
}
% Correzione caratteri speciali in listings
\lstset{
  literate={_}{{\_}}1
           {à}{{\`a}}1 {è}{{\`e}}1 {ì}{{\`i}}1
           {ò}{{\`o}}1 {ù}{{\`u}}1 {é}{{\'e}}1
}

% ================================================================================
% ALGORITHMS & PSEUDOCODE
% ================================================================================
\usepackage{algorithm}             % Ambiente per algoritmi float
\usepackage{algpseudocode}         % Pseudocodice stile “algorithmicx”

% ====================================================================================
% HEADINGS, CHAPTER & TOC STYLES
% ================================================================================
\usepackage{titlesec}              % Personalizza titoli di sezioni/capitoli
\titleformat{\chapter}[hang]{\normalfont\Huge\bfseries}{\thechapter.}{8pt}{\Huge\bfseries}
\titlespacing*{\chapter}{0pt}{-40pt}{5pt}
\titlespacing*{\section}{0pt}{0pt}{0pt}
\titlespacing*{\subsection}{0pt}{0pt}{0pt}

\usepackage[titles]{tocloft}       % Spazi nella ToC
\setlength{\cftbeforechapskip}{6pt}

% ================================================================================
% BIBLIOGRAPHY
% ================================================================================
\usepackage[
  backend=biber,
  sorting=none,
  style=numeric,
  maxbibnames=99
]{biblatex}                        % Gestione bibliografia
\addbibresource{refs.bib}          % File con le entry BibTeX

% ================================================================================
% MISCELLANEOUS UTILITIES
% ================================================================================
\usepackage{comment}               % Ambienti \begin{comment} ... \end{comment}
\usepackage{lipsum}                % Testo segnaposto \lipsum[1-2]

% ================================================================================
% END OF PACKAGE SETUP
% ================================================================================



\title{Inserire titolo}
\author{Giacomo Golino}
\date{Maggio 2025}

\newgeometry{lmargin=3cm,rmargin=3cm,tmargin=3cm,bmargin=3cm}   

\pagestyle{fancy} % Customizzazione testata e piè di pagina
\renewcommand{\chaptermark}[1]{\markboth{#1}{#1}}
\fancyhf{}
\fancyhead[L]{\thechapter. \leftmark}
\fancyfoot[C]{\thepage}

\usepackage{afterpage}

\newcommand\blankpage{%
    \null
    \thispagestyle{empty}%
    \addtocounter{page}{-1}%
    \newpage}

\begin{document}

 \begin{titlepage}
    \begin{figure}[h] % Logo Unibs
        \centering
        \includegraphics[width=72.4mm,height=30mm]{images/logo.png}
    \end{figure}

    \begin{center}
        \vspace{4mm} % Spazio verticale
        \LARGE{\uppercase{Dipartimento di Ingegneria dell'informazione}}\\
        \vspace{4mm}
        \LARGE{Corso di Laurea \\in Ingegneria Informatica Magistrale}
    \end{center}
    \begin{center}
        \LARGE{Relazione Finale}\\
        \LARGE{\textbf{Integrazione di funzionalità di Trigger-Action Programming nel Gemello Digitale di una Smart Home
}}
    \end{center}

    \begin{flushleft}
        \large
        \textbf{Relatore:}
        Prof.sa Daniela Fogli \\
        \textbf{Correlatore:}
        Prof.sa Barbara Barricelli \\
        \textbf{Correlatore:} Ing. Davide Guizzardi\\
    \end{flushleft}

    \begin{flushright}
        \large
        \textbf{Studente:}\\
        Giacomo Golino (719210)
    \end{flushright}

    \vspace*{\fill} % Spazio bianco per mettere il testo in fondo alla pagina
    \rule{0.8\textwidth}{0.6pt}\\ % Linea nera
    \centering{\Large{Anno Accademico 2024/2025}}
\end{titlepage}
\afterpage{\blankpage}
 %Frontespizio 
    %Permette di cambiare i margini delle pagine
     \hyphenpenalty=750 % Rende meno probabile lo spezzamento delle parole a capo
    \frontmatter
    \renewcommand\contentsname{Indice} % Indice
    \tableofcontents
     \setlength{\headheight}{14.49998pt}
 
\chapter{Introduzione}
\lipsum[1-2] % Introduzione
    \mainmatter
    \chapter{Titolo cap - 1}
\label{chap:uno}

\section{Introduzione}
Da introdurre gli obiettivi e il contesto della tesi, ed eventualmente la sua struttura.
    % --------------------------- INIZIO CAPITOLO --------------------------- %

\chapter{Stato dell'arte}
In questo capitolo verranno analizzati e approfonditi i \textbf{metodi} e le \textbf{applicazioni} per la creazione di \emph{automazioni} (chiamate anche \emph{routine}) in ecosistemi \emph{Internet of Things} (IoT).
In particolare, verranno affrontate con maggiore attenzione le \emph{smart home} come ecosistemi IoT, andando ad analizzare quali sono le principali piattaforme utilizzate per la creazione di \textit{automazioni}.

Si definiscono \emph{\textbf{automazioni}} o \emph{routine} degli insiemi strutturati di istruzioni che, una volta impostati, consentono di eseguire automaticamente una serie di azioni al verificarsi di determinati eventi (o \emph{trigger}). Tali azioni possono comprendere operazioni ripetitive, come l'accensione o lo spegnimento di dispositivi smart, oppure processi più complessi che coinvolgono diversi servizi, con l'obiettivo finale di semplificare e ottimizzare la gestione dell'ecosistema.

Le \emph{\textbf{azioni}} (\emph{action}) costituiscono la parte esecutiva di una routine: al verificarsi dell’evento scatenante (\emph{\textbf{trigger}}), l’azione definita viene avviata per compiere l’operazione desiderata. Può trattarsi di semplici attività oppure di processi articolati che si integrano con servizi diversi.

All’interno di questo capitolo verranno quindi presentati:
\begin{itemize}
  \item I concetti chiave relativi alle automazioni/routine e la loro importanza nel contesto IoT.
  \item Le principali piattaforme sul mercato che consentono di creare routine, nello specifico \textbf{Amazon Alexa}, \textbf{Google Home} e \textbf{Apple Casa}.
  \item Altri sistemi per il \textbf{Trigger-Action Programming} che si sono recentemente affermati e alcuni approcci proposti nella letteratura scientifica.

\end{itemize}


% -------------- SECTION 2.1 - CONCETTI BASE, AUTOMAZIONI E ROUTINE -------------- %


\newpage
\section{Concetti di Base}
Le \emph{automazioni} o \emph{routine} nel contesto dell’IoT consistono in regole di tipo \emph{trigger-action}, dove un evento scatenante (ad esempio, un orario predefinito, un comando vocale o una condizione ambientale rilevata da un sensore) provoca una o più azioni (come l’accensione di una luce, l’avvio di un elettrodomestico, l’invio di una notifica, ecc.). Questa logica di base, semplice da comprendere, si è rivelata estremamente potente e versatile in quanto:

\begin{itemize}
  \item Riduce la necessità di interventi manuali da parte dell’utente, automatizzando \textbf{task} ripetitivi.
  \item Consente di \emph{personalizzare} lo spazio domestico in base alle preferenze e alle abitudini di ognuno.
  \item Si presta a una gestione modulare: i vari eventi e azioni possono essere combinati per creare routine più avanzate.
\end{itemize}

Alcuni studi hanno evidenziato come la programmazione di tipo \emph{trigger-action} (TAP) si adatti alla maggior parte delle esigenze di automazione espresse dagli utenti \cite{Ur2014, Barricelli2024}, risultando intuitiva anche per chi non possiede competenze tecniche avanzate. In particolare, l’analisi di oltre 67.000 programmi condivisi su \href{https://www.IFTTT.com}{\texttt{IFTTT}} e un test di usabilità condotto su 226 partecipanti conferma che, grazie alla semplicità di combinare in modo flessibile molteplici \emph{trigger} e \emph{action}, la curva di apprendimento rimane bassa, favorendo un’ampia adozione \cite{Ur2014}.

Nell'immagine qui di seguito (fig. \ref{fig:ch2-trigger and actions}) vengono mostrati alcuni esempi di quelli che potrebbero essere dei \textbf{trigger} e delle \textbf{action}.

\begin{figure}[H]
    \centering
    \includegraphics[width=0.75\linewidth]{images/Chap - 2/Trigger and actions - no background.png}
    \caption{Esempio di trigger e azioni}
    \label{fig:ch2-trigger and actions}
\end{figure}


% --------------------------- SUBSECTION  2.1 - REQUISITI DI UNA BUONA AUTOMAZIONE --------------------------- %


\subsection{Requisiti di una buona automazione}
Per risultare efficace \cite{Reisinger2023SmartHomeReq}, un'automazione deve essere:
\begin{itemize}
  \item \textbf{Facile da creare e gestire}: l'utente finale (che spesso non possiede competenze di programmazione) deve poter definire e modificare le routine in maniera intuitiva.
  \item \textbf{Affidabile}: deve funzionare in modo consistente nel tempo, senza errori o interruzioni non previste.
  \item \textbf{Adattabile}: dev'essere in grado di gestire modifiche nelle preferenze dell’utente, nelle caratteristiche dei dispositivi o nell’assetto del sistema.
  \item \textbf{Sicura}: in un ecosistema connesso, la \emph{privacy} e la \emph{sicurezza} di reti e dispositivi sono fondamentali.
\end{itemize}

% --------------------------- SECTION 2.2 - PANORAMICA DELLE PIATTAFORME PRINCIPALI --------------------------- %


\section{Panoramica sulle piattaforme principali}
Storicamente, le automazioni si basavano principalmente su \emph{timer} o sensori semplici (come termostati o rilevatori di movimento) per poter funzionare. 
Al giorno d'oggi, invece, le piattaforme commerciali (e non) offrono soluzioni ben diverse rispetto a quelle di una volta. 
Nello specifico attualmente vengono utilizzate:
\begin{itemize}
  \item Interfacce grafiche \emph{drag-and-drop} per comporre in modo visivo le regole.
  \item Integrazione con \emph{assistenti vocali} (es. Amazon Alexa, Google Assistant e Apple Siri) per impostare routine con frasi naturali.
  \item Utilizzo di tecniche di \emph{machine learning} e \emph{context awareness} per proporre automazioni “intelligenti— o per adattare le routine al comportamento dell’utente.
\end{itemize}

Il panorama degli strumenti per la creazione di automazioni è estremamente ampio. In particolare, i sistemi \textbf{Amazon Alexa}, \textbf{Google Home} e \textbf{Apple Casa} sono le principali piattaforme commerciali per la gestione domestica, ciascuno caratterizzato da specifiche architetture, protocolli di comunicazione e modelli di integrazione con dispositivi di terze parti.

% --------------------------- SUBSECTION 2.2.1 - ALEXA --------------------------- %


\subsection{Amazon Alexa}
Amazon Alexa è un assistente vocale lanciato inizialmente su dispositivi \emph{Echo}, poi rapidamente esteso a numerosi dispositivi di terze parti. Le \emph{routine} di Alexa possono essere create tramite:
\begin{itemize}
  \item L’app Alexa su smartphone.
  \item Interazione vocale diretta (nello specifico grazie alla sottoscrizione del servizio \textit{Alexa+}, disponibile al momento solo negli USA).
  \item \emph{Skill} aggiuntive, sviluppate da terze parti.
\end{itemize}

Inoltre, negli ultimi anni, Amazon ha introdotto:
\begin{itemize}
  \item \textbf{Riconoscimento di suoni specifici}: ad esempio, la piattaforma è in grado di distinguere il suono di vetri rotti, avvisando di conseguenza l'utente. È importante sottolineare che questa funzionalità è disponibile al momento solamente su \textbf{Echo smart speakers} e sui dispotivi \textbf{Echo smart displays}.\\
  
  \item \textbf{Hunches}: questa funzionalità, se attivata attraverso l’applicazione, consente ad Alexa di apprendere determinate routine ricorrenti, così da poter avvisare l’utente qualora tali azioni non vengano eseguite.
  Ad esempio, Alexa potrebbe imparare che ogni volta che un utente esce di casa, è solito spegnere tutte le luci e chiudere la porta. Nel caso in cui l'utente dovesse dimenticare di svolgere una o entrambe le azioni, il sistema lo notificherebbe, chiedendo se è necessario eseguire queste azioni.\\
  
  \item \textbf{Integrazioni con servizi esterni}: questa funzionalità consente l'integrazione con servizi di terze parti come calendari, servizi di musica in streaming o applicazioni di messaggistica.
\end{itemize}


% --------------------------- SUBSECTION 2.2.2 - GOOGLE HOME --------------------------- %


\subsection{Google Home}
Google Home si basa sull’assistente vocale Google Assistant, sfruttando l’ampia rete di servizi Google. Le automazioni possono derivare dall’utilizzo di:
\begin{itemize}
  \item Comandi vocali per la creazione e gestione delle \emph{routine}. Gli utenti possono avviare la configurazione di una routine semplicemente pronunciando frasi come “Ok Google, crea una routine per...—, dopodiché Google Assistant guiderà attraverso i passaggi per impostare un \emph{trigger} e una o più \emph{action}. Questo metodo semplifica notevolmente l’automazione per chi preferisce un'interazione naturale senza dover passare obbligatoriamente da un'applicazione.
  \item L’app Google Home, che permette di definire \emph{trigger} e \emph{action} utilizzando dispositivi compatibili.
  \item L’integrazione con Google Calendar o Google Maps (permettendo, ad esempio, di attivare una routine se si è vicini a casa).
\end{itemize}


L’ecosistema Google si è evoluto introducendo:
\begin{itemize}
  \item \textbf{Eventi geolocalizzati (geofencing)}: la routine si attiva o disattiva in base alla posizione dell’utente. Il \emph{geofencing} è una tecnologia basata sulla geolocalizzazione che permette di definire un'area virtuale attorno a una posizione fisica (come casa ad esempio). Quando il dispositivo dell’utente entra o esce da questa zona predefinita, viene attivata una routine automatica. Ad esempio, si può configurare l’accensione delle luci smart quando si arriva a casa o la disattivazione del riscaldamento quando si lascia l’abitazione.
  \item \textbf{Riconoscimento vocale avanzato (Voice Match)}: l’assistente riconosce la voce di diverse persone e personalizza alcune automazioni (ad esempio, la musica preferita) in base al profilo.
\end{itemize}


% --------------------------- SUBSECTION 2.2.1 - APPLE CASA --------------------------- %


\subsection{Apple Casa}
Apple Casa è l’applicazione di Apple per la gestione della domotica, basata sulla piattaforma HomeKit. Rispetto a soluzioni più aperte, punta molto sulla semplicità d’uso e su elevati standard di privacy e sicurezza.
Di seguito sono riportate alcune delle principali caratteristiche di questa piattaforma:
\begin{itemize}
  \item \textbf{App Casa su iOS}: qui si impostano automazioni basate sull’orario, sul rilevamento di un sensore, o sull’entrata/uscita da una determinata area geografica.
  \item \textbf{Supporto a scene e stanze}: l’utente può creare “scene— (sequenze di azioni) o associare dispositivi a stanze, semplificando la gestione.
  \item \textbf{Apertura verso standard di connettività}: Apple ha recentemente esteso il supporto a protocolli come Matter, garantendo più integrazione con prodotti non-Apple.
\end{itemize}

Le ultime versioni di iOS hanno introdotto:
\begin{itemize}
  \item \textbf{Rilevamento della presenza via Apple Watch}: se l’utente indossa un Apple Watch, il sistema può capire se si trova effettivamente in casa (o nei dintorni).
\end{itemize}


% --------------------------- SECTION 2.3 - METODI E APPLICAZIONI PER LA CREAZIONE DI AUTOMAZIONI --------------------------- %


\section{Metodi e Applicazioni per la Creazione di Automazioni}
Oltre alle piattaforme citate, esistono molteplici approcci al \emph{Trigger-Action Programming}, come visto in \cite{Barricelli2024, Corno2021, Barricelli2022MultiModal}.
Nei seguenti sottocapitoli verrà proposta un’analisi delle caratteristiche principali delle piattaforme più famose che permettono la \emph{Trigger-Action Programming}.


% --------------------------- SUBSECTION - 2.3.1 - Applicazioni per il Trigger-Action Programming --------------------------- %
\subsection{Applicazioni per il Trigger-Action Programming}

\subsubsection{IFTTT}
Acronimo di \textbf{If This Then That} è una delle prime piattaforme di automazione su larga scala, nota per la sua interfaccia intuitiva e la capacità di integrare servizi eterogenei senza richiedere competenze tecniche avanzate \cite{Ur2014}.
È possibile lavorare con questa piattaforma direttamente dal web, accedendo al sito: \href{https://ifttt.com}{\texttt{ifttt.com}}.\\
Gli utenti possono creare semplici regole \emph{IF-THEN} (se accade un evento (\textit{IF}), allora esegui un’azione (\textit{THEN})) selezionando trigger e azioni predefinite. Ad esempio, si può configurare l’invio automatico di un'email quando viene pubblicato un nuovo post su un blog. Tuttavia, rispetto ad altre soluzioni, IFTTT presenta limitazioni in termini di personalizzazione e controllo, e alcune funzionalità avanzate sono disponibili solo nella versione a pagamento.\\ In (fig. \ref{fig:ch2_ifttt_ex}) viene mostrato un esempio della creazione di un'automazione che, al verificarsi di una specifica condizione meteo (\textit{trigger}) invia una notifica all'utente (\textit{action}). \\

\begin{figure}[H]
    \centering
    \includegraphics[width=0.8\linewidth]{images/Chap - 2/ifttt example.png}
    \caption{Esempio di creazione di una automazione con l'ausilio di IFTTT}
    \label{fig:ch2_ifttt_ex}
\end{figure}


% --------------------------- SUBSUBSECTION - HOME ASSISTANT --------------------------- %


\subsubsection{Home Assistant}
È una piattaforma open-source per la gestione avanzata della domotica, particolarmente apprezzata dagli utenti più esperti per la sua elevata configurabilità e il supporto a un’ampia gamma di dispositivi. 
È possibile reperire la documentazione di questa piattaforma sul sito:  \href{https://www.home-assistant.io}{\texttt{www.home-assistant.io}}.

Le automazioni possono essere create sia tramite un’interfaccia grafica (Graphic User Interface, GUI) che mediante codice YAML, consentendo logiche più sofisticate rispetto a soluzioni come IFTTT. 
Nell'esempio riportato in (fig. \ref{fig:ch2:home_assistant_gui}), è possibile vedere come è stato possibile programmare una sequenza di eventi condizionali, come l'accensione delle luci in una particolare stanza quando un sensore di un garage rileva l'apertura del suddetto, solo se il sole è già calato, mediante l'interfaccia grafica di Home Assistant.

\begin{figure}[H]
    \centering
    \includegraphics[width=1\linewidth]{images/Chap - 2/homeassistant_gui.png}
    \caption{Esempio di creazione di un'automazione mediante GUI}
    \label{fig:ch2:home_assistant_gui}
\end{figure}

La creazione di automazioni mediante codice YAML può risultare però più complessa per degli utenti non esperti, rispetto a quella tramite GUI.
Nell’esempio di seguito riportato in (fig. \ref{fig:ch2_homeassistant_yaml}) è illustrato come sia possibile definire un'automazione basata sulla geolocalizzazione degli utenti. In particolare, l'evento scatenante (\texttt{trigger}) si verifica quando uno dei dispositivi tracciati cambia stato da \texttt{not\_home} a \texttt{home}, attivando così le azioni corrispondenti dopo un ritardo di un minuto.

\begin{figure}[!ht]
    \centering
    \includegraphics[width=1\linewidth]{images/Chap - 2/homeassistant_yaml.png}
    \caption{Esempio di codice YAML per la creazione di un'automazione}
    \label{fig:ch2_homeassistant_yaml}
\end{figure}


% --------------------------- SUBSECTION 2.3.3 - NODE-RED --------------------------- %


\subsubsection{Node-RED} \label{Node-RED}
È un ambiente di sviluppo visuale basato su flussi, che consente di creare automazioni collegando elementi modulari chiamati “nodi—.
Ogni nodo rappresenta un trigger, un’elaborazione o un’azione, e può essere connesso agli altri in un diagramma di flusso grafico. Questo approccio offre un controllo più avanzato rispetto a IFTTT e una maggiore semplicità rispetto alla configurazione testuale di Home Assistant. Ad esempio, un nodo può ricevere dati da un sensore di temperatura, elaborarli con una soglia personalizzata e inviare una notifica solo se viene superato un valore critico. Tuttavia, essendo una piattaforma più tecnica, richiede una certa familiarità con concetti di programmazione e rete.

È possibile reperire la documentazione per l'utilizzo di questa piattaforma presso il sito: \href{https://nodered.org/docs/}{\texttt{www.nodered.org}}.
In (fig. \ref{fig:ch2_redNode}) si mostra un flusso di Node-RED che, dopo aver recuperato le condizioni meteorologiche e la temperatura corrente, interroga uno smartwatch per rilevare il numero di passi compiuti dall'utente durante la giornata; se tale valore è inferiore a $1000$, il sistema invia automaticamente un'e-mail di notifica. \\

\begin{figure}[H]
    \centering
    \includegraphics[width=1\linewidth]{images/Chap - 2/rednode 2.png}
    \caption{Esempio di generazione di un'automazione con Node-RED}
      \label{fig:ch2_redNode}
  \end{figure}

  % \begin{figure}[H]
  %     \centering
  %     \includegraphics[width=1\linewidth]{images/Chap - 2/RedNote.png}
  %     \caption{Esempio di generazione di un'automazione con Node-RED}
  %     \label{fig:ch2_redNote}
  % \end{figure}


% --------------------------- SUBSECTION 2.3.2 - CONFRONTO FRA LE TRE PIATTAFORME --------------------------- %


\subsection{Confronto fra le tre piattaforme}
\label{sec:confronto_piattaforme}

Per confrontare IFTTT, Home Assistant e Node-RED sono state considerate metriche analizzate in \cite{McCall2023,Thomas2023,Aavild2024} come:

\begin{enumerate}[label=\alph*)]
  \item \textbf{Facilità d’uso e curva di apprendimento}
  \item \textbf{Potenza espressiva delle regole}
  \item \textbf{Modalità di esecuzione (cloud o locale)}
  \item \textbf{Ampiezza delle integrazioni disponibili}
  \item \textbf{Costo di adozione}
\end{enumerate}

Studi recenti sottolineano come gli strumenti interamente visuali (es.\ IFTTT) favoriscano la rapidità di prototipazione, ma impongano limiti man mano che cresce la complessità logica \cite{McCall2023}. Soluzioni a flusso come Node-RED colmano questo gap offrendo un controllo granulare senza richiedere la completa transizione a un approccio testuale; Home Assistant si colloca a metà strada grazie alla combinazione di un'interfaccia grafica con la possibilità dell'utilizzo di YAML, mantenendo una soglia di ingresso relativamente accessibile ma consentendo automazioni avanzate \cite{Thomas2023,Aavild2024}.

\definecolor{lightgray}{HTML}{b0b7c2}
\rowcolors{2}{white}{lightgray}           % righe alternate


\begin{table}[H]
  \centering
  \renewcommand{\arraystretch}{1.4}  % spazio verticale extra (facoltativo)
  \rowcolors{2}{white}{lightgray}   % se vuoi righe alternate
  \begin{tabular}{|C{0.20\textwidth}|C{0.2\textwidth}|C{0.25\textwidth}|C{0.2\textwidth}|}
    \hline
    \textbf{Criterio} & \textbf{IFTTT} & \textbf{Home Assistant} & \textbf{Node-RED} \\ \hline
    \textbf{Facilità d’uso}    & Alta                   & Media           & Media-bassa \\ \hline
    \textbf{Potenza logica}    & Bassa –– singola regola      & Alta –– automazioni / script & Alta –– flussi arbitrari \\ \hline
    \textbf{Esecuzione}        & Cloud                        & Locale / Cloud              & Locale \\ \hline
    \textbf{Integrazioni}      & $\sim$ 700 servizi cloud      & $\sim$ 2500 integrazioni      & $\sim$ 3500 nodi \\ \hline
    \textbf{Costo}             & Freemium                     & Open-source                  & Open-source \\ \hline
  \end{tabular}
  \caption{Confronto fra IFTTT, Home Assistant e Node-RED}
  \label{tab:confronto_piattaforme}
\end{table}



In sintesi, \emph{IFTTT} si distingue per la rapidità con cui consente di realizzare automazioni semplici; \emph{Home Assistant} eccelle nei casi domestici complessi grazie all’ampio catalogo di integrazioni e alla combinazione di interfaccia grafica e configurazioni testuali; \emph{Node-RED} offre la massima flessibilità architetturale, risultando ideale per scenari IoT multi-dominio che richiedono logiche di flusso elaborate.



% --------------------------- SUBSECTION 2.3.5 - APPROCCI NELLA LETTERATURA SCIENTIFICA --------------------------- %


\subsection{Approcci nella letteratura scientifica}
Negli ultimi anni, la ricerca si è concentrata su come rendere accessibile la creazione di routine e flussi di automazione anche a quegli utenti che non possiedono competenze di programmazione (\textit{End-User Development}, EUD). In particolare, sono emersi due approcci complementari per aiutare l’utente nella definizione di regole e workflow: le \emph{interfacce visuali} e le \emph{interfacce conversazionali} \cite{Barricelli2024}. \\
In seguito si è poi deciso di provare a combinare i punti di forza di entrambi gli approcci dando vita alle \emph{interfacce multimodali}, che consentono di sfruttare simultaneamente modalità di input e output sia visuali sia conversazionali, offrendo così un’esperienza più ricca e flessibile per l’utente.


% --------------------------- SUBSECTION 2.3.6 - INTERFACCE VISUALI --------------------------- %

\subsubsection{Interfacce Visuali}

Le interfacce visuali hanno come principale vantaggio la rappresentazione esplicita dei \emph{trigger} e delle \emph{action}, tramite blocchi e connettori, o attraverso grafiche facilmente riconoscibili dall’utente finale. Paradigmi come il \emph{flow-based programming} e il \emph{block-based programming} permettono di trascinare e collegare componenti, rendendo chiari i passaggi di input e output e minimizzando gli errori di configurazione. Inoltre, rappresentazioni iconiche (ad esempio una lampadina o un termostato) possono semplificare l’interazione e offrire un riscontro immediato di ciò che si sta progettando.

Un semplice esempio per la creazione di un'automazione mediante \textit{interfaccia visuale} è possibile vederla in (fig. \ref{fig:ch2_redNode}).

\paragraph{Vantaggi}

Gli editor grafici offrono una \emph{bassa soglia di ingresso} per gli utenti non programmatori: negli studi controllati il tempo di apprendimento medio è inferiore a 15 minuti \cite{Desolda2017}. 
Nel progetto, gli utenti coinvolti hanno definito la programmazione come «estremamente facile», segnalando un immediato senso di successo nella creazione di regole semplici~\cite{Coutaz2016}.


\paragraph{Svantaggi.}
All’aumentare della complessità, i benefici iniziali si attenuano: l’espressione di condizioni composte o di regole multi‑dispositivo fa crescere tempi di completamento e tassi d’errore~\cite{Coutaz2016}. \\
Ambienti di sviluppo come Node‑RED (vedasi \ref{Node-RED})  mostrano problemi di \emph{scalabilità visiva}: poche decine di nodi superano lo spazio dello schermo, generando sovraccarico cognitivo e rendendo difficile comprendere le cause di un’azione \cite{Lago2021}.
Ulteriori limiti emersi riguardano:
\begin{itemize}
    \item L’ambiguità tra eventi e stati, che confonde gli utenti~\cite{Lago2021}.
    \item La mancanza di supporto multi‑utente e di possibilità di simulazione, evidenziata da Caivano et al. nel confronto Atooma/IFTTT/Tasker \cite{Caivano2018}.
    \item L’assenza di meccanismi di risoluzione di conflitti: loop, collisioni e ridondanze possono produrre comportamenti inattesi \cite{Corno2020,Chen2021}.
\end{itemize}
In sintesi, le interfacce visuali rimangono efficaci per regole semplici e utenti alle prime armi, ma necessitano di funzioni di astrazione, filtraggio e analisi per restare usabili in scenari reali e dinamici.


% --------------------------- SUBSECTION 2.3.7 - INTERFACCE CONVERSAZIONALI --------------------------- %


\subsubsection{Interfacce Conversazionali}
Parallelamente all’approccio visuale, diversi studi hanno approfondito l’efficacia delle cosiddette \emph{interfacce conversazionali} \cite{Barricelli2024}, che consentono la definizione di routine mediante interazione in linguaggio naturale con assistenti vocali (ad es.\ Amazon Alexa, Google Assistant, Apple Siri). 
In questo caso, la creazione di una regola o routine avviene attraverso un dialogo guidato, dove l’utente può specificare trigger e azioni utilizzando frasi colloquiali \cite{Barricelli2024, Gallo2024Conversational}.

L’utilizzo del linguaggio naturale può apparire più intuitivo rispetto a un’interfaccia a blocchi, specialmente per utenti poco esperti, poiché non richiede di apprendere una sintassi formale o di manipolare elementi grafici. Tuttavia, la sfida maggiore risiede nella corretta interpretazione delle richieste (\emph{Natural Language Understanding, NLU}) e nella necessità di meccanismi di conferma o riepilogo, che riducano il rischio di errori o ambiguità.

Viene riportato di seguito un esempio di un'interazione vocale creata partendo dal sistema analizzato in \cite{Barricelli2024}. 

\noindent \textbf{Esempio di una possibile interazione vocale:}
\begin{quote}
\textbf{Utente}: “Alexa, voglio creare una nuova routine.—\\
\textbf{Assistente}: “Va bene, come desideri attivare la routine? Puoi impostare un comando vocale, un giorno e orario specifici, oppure l’alba o il tramonto.—\\
\textbf{Utente}: “Imposta il comando vocale.—\\
\textbf{Assistente}: “Hai scelto di attivare la routine tramite un comando vocale. Dimmi la frase che vuoi usare per avviarla.—\\
\textbf{Utente}: “Accendi luci salotto.—\\
\textbf{Assistente}: “Perfetto, hai impostato il comando vocale ‘Accendi luci salotto.’ Ora, desideri aggiungere una o più azioni alla routine?—\\
\textbf{Utente}: “Sì, aggiungi il comando di accendere le luci del salotto.—\\
\textbf{Assistente}: “Aggiungo l’azione ‘accensione luci salotto.’ Vuoi aggiungere altro o salvare la routine?—\\
\textbf{Utente}: “Salva routine.—\\
\textbf{Assistente}: “La tua routine è stata salvata con il comando vocale ‘Accendi luci salotto.’ Puoi modificarla in qualsiasi momento.—
\end{quote}

\noindent In questo esempio, si evidenzia come l'assistente debba fornire indicazioni chiare per ciascun passaggio, ripetere o confermare le scelte dell'utente e limitare la complessità di ogni richiesta per evitare sovraccarico cognitivo. Questo approccio è coerente con i principi di progettazione conversazionale e con la cosiddetta \emph{Miller’s Law} \cite{Miller1994}, che evidenzia come la capacità di memorizzazione e comprensione degli utenti sia limitata a un numero ristretto di elementi per volta.

\paragraph{Vantaggi}
\noindent
\newline \newline L’interazione in linguaggio naturale consente un accesso ``senza mani'' e a bassa soglia cognitiva: negli esperimenti con smart speaker \emph{speech‑only}, il tasso di completamento delle routine semplici ha raggiunto il 94\,\% con un punteggio SUS medio di 78/100 ~\cite{Barricelli2024, Lago2021, Brooke1996} .  
Jarvis (assistente conversazionale progettato per la gestione di sistemi IoT complessi tramite linguaggio naturale) \cite{Lago2021} mostra che la formulazione vocale riduce il tempo medio di espressione di una regola del 30\,\% rispetto alla costruzione grafica in Node‑RED (\ref{Node-RED}) \cite{Lago2021}.  

L’impiego di chatbot basati su modelli linguistici di grandi dimensioni (es.\ RuleBot++) migliora la comprensione da parte del sistema delle richieste effettuate dall'utente \cite{Gallo2024Conversational}.  

Inoltre gli assistenti vocali risultano inclusivi per utenti con ridotta capacità visiva o con mani occupate, ampliando l'accessibilità del controllo domestico~\cite{Sciuto2018}.

\paragraph{Svantaggi} 
\begin{itemize}
  \item \textbf{Ambiguità semantiche}: la stessa frase può essere interpretata in modi diversi; nelle prove di Jarvis oltre il 20\,\% dei comandi liberi ha richiesto chiarimenti \cite{Lago2021}.
  \item \textbf{Recupero da errori}: senza un supporto visivo, gli utenti faticano a capire che cosa non sia stato compreso e come riformulare; ParlAmI riporta interazioni interrotte nel 15\,\% dei casi~\cite{ParlAmi2019}.
  \item \textbf{Scalabilità logica}: con regole che includono più di due trigger e tre azioni aumentano i turni di dialogo; RuleBot++ registra un +42\,\% di turni di chiarimento rispetto a regole elementari \cite{Gallo2024Conversational}.
  \item \textbf{Privacy e fiducia}: nei diari d’uso longitudinali di Sciuto et al.\ il 38\,\% degli utenti ha limitato i comandi sensibili per timore di ascolto continuo~\cite{Sciuto2018}.
\end{itemize}

In sintesi, le interfacce conversazionali semplificano l'accesso e rendono naturale la definizione di automazioni semplici; per scenari complessi occorrono strategie di disambiguazione, conferma e sintesi per mantenere affidabilità ed efficienza.


% --------------------------- SUBSECTION 2.3.7 - INTERFACCE MULTIMODALI --------------------------- %

\subsubsection{Interfacce Multi-modali}
Per far fronte alle problematiche che possono nascere dall'utilizzo di un'interfaccia puramente conversazionale o visuale è possibile andare ad utilizzare quelle che vengono definite \textit{interfacce multi-modali}. Con questo termine si identificano quei dispositivi che permettono di interagire con il sistema sfruttando simultaneamente più canali comunicativi, come ad esempio voce e visualizzazione di un'interfaccia grafica. \cite{Barricelli2022MultiModal, ParlAmi2019}

Un’evoluzione di questo filone è \textbf{RuleBot ++}
\cite{Gallo2024Conversational}, un chatbot basato su ChatGPT (GPT-4)
integrato in un’interfaccia multi-modale:  
l’utente può impartire i comandi a voce o via chat,
visualizzare la regola generata sul display (o in app) e,
se necessario, correggerla toccando i singoli parametri.
Nei test di usabilità (16 partecipanti) RuleBot ++ ha fatto
registrare un punteggio SUS medio di 78/100 e un
\emph{time-on-task} inferiore del 35 \% rispetto alla GUI
di Home Assistant. \cite{Brooke1996}

In (fig. \ref{fig:chap2 multimodal}) è possibile vedere un ulteriore esempio di interfaccia multi-modale utilizzata per la definizione di una nuova routine tramite l'ausilio di un dispositivo \textit{Amazon Alexa}. \cite{Barricelli2022MultiModal}

\begin{figure}[H]
    \centering
    \includegraphics[width=1\linewidth]{images/Chap - 2/Multimodal.png}
    \caption{Esempio di un sistema multi-modale utilizzato per la generazione di routine}
    \label{fig:chap2 multimodal}
\end{figure}

Una recente analisi sperimentale condotta in \cite{Barricelli2022MultiModal} ha evidenziato i vantaggi dell'utilizzo di soluzioni multi-modali per la creazione di routine all'interno di ecosistemi IoT, in quanto: 
\begin{itemize}
\item Le interfacce multi-modali risultano più apprezzate dagli utenti rispetto alle interazioni \emph{voice-only}, specialmente per la possibilità di visualizzare in tempo reale le scelte già effettuate e selezionare i comandi su schermo.
\item Le interfacce conversazionali puramente vocali mantengono un elevato grado di accessibilità, ma richiedono una maggiore attenzione e memoria da parte dell’utente per seguire l’intero dialogo ed evitare di perdersi tra le opzioni disponibili.
\item Aspetti come la dipendenza dal brand (ad esempio, la familiarità con Alexa o con Google Assistant) e il livello di esperienza con i dispositivi smart possono influire significativamente sulla percezione di usabilità e soddisfazione, suggerendo che la personalizzazione dell’interfaccia e la coerenza del dialogo siano fattori chiave.
\end{itemize}

\paragraph{Vantaggi}
\begin{itemize}
  \item \textbf{Integrazione voce e display}: quando input vocali e riscontri grafici/tattili convivono, i
        limiti del canale singolo si riducono sensibilmente: l’utente
        può parlare per avviare la routine e, nello stesso tempo,
        verificare a colpo d’occhio ciò che il sistema ha compreso.

  \item \textbf{Maggiore usabilità del sistema}: come mostrato in \cite{Barricelli2022MultiModal}, l'utilizzo di una soluzione multimodale ha portato
        ad un incremento del tasso di completamento dei compiti assegnati agli utenti. Nello specifico si è passati da un 88 \% (solo voce) ad un 97 \% dei task completati e ad innalzo del punteggio SUS da 74/100 a 83/100. Tutto ciò conferma
        che la doppia modalità migliora sia efficacia sia
        soddisfazione. \cite{Brooke1996}

  \item \textbf{Maggiore controllo}: i partecipanti allo studio hanno apprezzato di poter usare il
        display per rivedere o correggere parametri complessi,
        lasciando alla voce i comandi rapidi («Alexa, salva»).\cite{Barricelli2022MultiModal}

  \item \textbf{Accessibilità amplificata}: la ridondanza dei canali favorisce pubblici diversi:  
        persone con disabilità visive possono affidarsi alla sintesi
        vocale, mentre chi opera in ambienti rumorosi ricorre all’interfaccia touch.

  \item \textbf{Supporto “intelligente— alla correzione}: integrando un \textit{Large Language Model} (LLM), come nel caso di \emph{RuleBot ++},
        l’interfaccia è in grado di proporre suggerimenti o versioni
        alternative della routine, riducendo errori sintattici e
        velocizzando la rifinitura delle regole \cite{Gallo2024Conversational}.
\end{itemize}

\paragraph{Svantaggi}
\begin{itemize}
  \item \textbf{Aumento della complessità dell’UI}: alcuni utenti hanno segnalato   confusione sul “dove guardare— quando la creazione di una routine visualizzava contemporaneamente conferme testuali e vocali~\cite{Barricelli2024}.
  \item \textbf{Costi e frammentazione}: display intelligenti e robot aumentano i costi
        di adozione e introducono differenze tra ecosistemi (Amazon vs Google)
        che richiedono percorsi di progettazione paralleli~\cite{Barricelli2024}.
  \item \textbf{Gestione dei conflitti multimodali}: ParlAmI riporta episodi
        di competizione tra input vocali e tocchi simultanei, con necessità di
        strategie di arbitraggio per evitare comandi duplicati~\cite{ParlAmi2019}.
  \item \textbf{Maggiore carico cognitivo in scenari complessi}: quando la
        regola supera tre azioni, il passaggio continuo voce a touch e viceversa aumenta il tempo
        medio di completamento del 22 \% rispetto alla sola interazione touch~\cite{Barricelli2024}.
\end{itemize}

    \chapter{Architettura del sistema}
\label{chap:digital-twin-smart-home}

In questo capitolo verrà descritta e approfondita l'architettura del sistema oggetto dell'elaborato. 

Il \emph{Gemello Digitale} in questione è progettato per la gestione intelligente di abitazioni, con una particolare attenzione alla sostenibilità energetica e all'ottimizzazione dei consumi.

\section{Introduzione all'architettura}
L'architettura è concepita per favorire una gestione ottimizzata e sostenibile delle risorse energetiche e degli apparati domestici. Uno dei principali obiettivi del sistema è la creazione di un'interazione semplice e intuitiva tra l'utente e l'ambiente domestico intelligente.

Una rappresentazione grafica dettagliata di tale architettura viene mostrata in (fig. \ref{fig:arch_overview}).

\begin{figure}[H]
    \centering
    \includegraphics[width=1\linewidth]{images/Chap - 3/architecture.png}
    \caption{Rappresentazione dell'architettura presa in analisi}
    \label{fig:arch_overview}
\end{figure}

Tale architettura consente agli utenti di realizzare automazioni domestiche attraverso un'interfaccia grafica (GUI). Questa interfaccia, sulla base di controlli e previsioni relativi ai consumi energetici, restituisce un feedback all'utente indicando chiaramente se l'automazione proposta è consentita oppure se genera conflitti con altre automazioni precedentemente impostate o, ancora, se rischia di eccedere il limite massimo di consumo energetico, andando a suggerire possibili modifiche all'automazione che si stava cercando di creare.

\newpage
\section{Digital Twin Interface}
La \textit{Digital Twin Interface} costituisce la componente del sistema che permette
all'utente finale di interagire con il Gemello Digitale sviluppato in questa tesi.
A differenza della \textit{Automation Creation App} conversazionale progettata dal CNR
e attualmente eseguita come applicazione esterna, la Digital Twin Interface espone una
applicazione web tradizionale che offre viste analitiche e strumenti operativi per
comprendere lo stato della smart home, controllarne i dispositivi e integrare le nuove
funzionalit\`a di generazione delle automazioni descritte nel capitolo successivo.

L'interfaccia attuale \`e organizzata in quattro sezioni principali:
\begin{itemize}
    \item \textbf{Dashboard} \textemdash{} riassume lo stato energetico della casa mostrando
    mappe interattive degli ambienti, indicatori sui dispositivi attivi e l'esito dei
    suggerimenti energetici ricevuti dal \textit{Data Analysis Module}. Da questa vista
    l'utente pu\`o inviare comandi manuali che vengono inoltrati al \textit{Simulation
    Management Module} (fig. \ref{fig:dig_twin_to_simul}).
    \item \textbf{Consumption} \textemdash{} espone grafici storici e predittivi dei consumi
    energetici (fig. \ref{fig:dam}) alimentati dalle API del \textit{Data Analysis Module}.
    L'utente pu\`o cambiare intervalli temporali, confrontare dispositivi e ricevere
    notifiche quando le stime superano le soglie definite nel \textit{Configuration
    Management Module}.
    \item \textbf{Automations} \textemdash{} elenca le automazioni presenti in Home Assistant,
    permettendo di attivarle o disattivarle, avviarne l'esecuzione e consultare i
    suggerimenti prodotti dal Simulation Management Module quando vengono rilevati conflitti
    o consumi eccessivi (fig. \ref{fig:sim_to_dig}). In questa sezione verr\`a integrato il
    nuovo builder visuale implementato in questa tesi, eliminando la dipendenza dalla sola
    Automation Creation App.
    \item \textbf{Configuration e User area} \textemdash{} permettono di modificare i dati
    personali, impostare le preferenze dell'utente e configurare mappe, dispositivi,
    gruppi e piano energetico. I dati inseriti vengono salvati nel \textit{Configuration
    Management Module} e resi disponibili agli altri moduli del sistema.
\end{itemize}

Oltre alle funzionalit\`a rivolte all'utente, la Digital Twin Interface gestisce la
comunicazione con Home Assistant, con il Simulation Management Module e con i servizi Rulebot:
\begin{itemize}
    \item invia i comandi manuali e le nuove automazioni al Simulation Management Module,
    che dopo la simulazione li inoltra al \textit{Home Assistant Integration Module} per
    l'esecuzione effettiva (fig. \ref{fig:sim_to_home});
    \item sincronizza lo stato dei dispositivi e delle automazioni leggendo i dati esposti
    dall'Home Assistant Integration Module e li rende disponibili alla pagina Automations;
    \item integra nella stessa interfaccia la vista dell'applicazione Rulebot esistente,
    fornendo all'utente un flusso continuo fra generazione, simulazione e attuazione
    delle regole.
\end{itemize}

La Automation Creation App rimane oggi un canale complementare per gli utenti che prediligono
l'interazione conversazionale e continua a essere eseguita come componente esterna.
L'obiettivo di questa tesi \`e integrare progressivamente quelle funzionalit\`a nella
Digital Twin Interface, rendendo il builder visuale il punto di accesso unico alla creazione
delle automazioni: il flusso sar\`a interamente web, dalla definizione della regola
all'invio ai moduli di simulazione e all'implementazione su Home Assistant.

In sintesi, la Digital Twin Interface \`e il punto di contatto tra utente e architettura
software e verr\`a ulteriormente estesa nel capitolo successivo per completare il percorso
di progettazione e attivazione delle automazioni.


\section{Simulation Management Module}

Il \emph{Simulation Management Module} rappresenta un componente centrale nell'architettura del gemello digitale di una smart home, svolgendo un ruolo chiave nel garantire che le automazioni proposte dagli utenti siano efficaci, sicure e coerenti con le preferenze personali e gli obiettivi di ciascun utente. 

Questo modulo gestisce principalmente la simulazione e la validazione preventiva delle automazioni definite dagli utenti, al fine di prevenire eventuali conflitti e massimizzare l'efficienza operativa della smart home.

Nello specifico il processo inizia quando l'utente definisce una nuova automazione, la quale viene inviata direttamente al \textit{Simulation Management Module} (fig. \ref{fig:user_to_simManMod}) \\[0.4cm]

\begin{figure}[H]
    \centering
    \includegraphics[width=0.3\linewidth]{images/Chap - 3/Simulation/simulation 1.png}
    \caption{Invio delle automazioni dall'utente al Simulation Management Module}
    \label{fig:user_to_simManMod}
\end{figure}

A questo punto, il modulo procede ad acquisire dal \emph{Home Assistant Integration Module} le automazioni che son già state definite all'interno della Smart Home, insieme ai dati relativi agli stati (come consumi, temperature, ecc.) di tutti i dispositivi collegati. 

È possibile visualizzare la porzione di architettura che si occupa di queste operazioni in (fig. \ref{fig:home_to_simulation}) \\[0.4cm]

\begin{figure}[H]
    \centering
    \includegraphics[width=0.3\linewidth]{images/Chap - 3/Simulation/simulation 2.png}
    \caption{Acquisizione da parte del Simulation Management Module delle automazioni e dei dati della Smart Home}
    \label{fig:home_to_simulation}
\end{figure}

Parallelamente, il \textit{Simulation Management Module} procede ad acquisire tutte le informazioni sugli obiettivi e le preferenze impostate dall'utente tramite il \textit{Configuration Management Module}, garantendo così il rispetto delle sue esigenze personali (fig. \ref{fig:conf_to_sim}). \\[0.4cm]

\begin{figure}[H]
    \centering
    \includegraphics[width=0.8\linewidth]{images/Chap - 3/Simulation/simulation 3.png}
    \caption{Acquisizione da parte del Simulation Management Module delle preferenze dell'utente}
    \label{fig:conf_to_sim}
\end{figure}

Una volta raccolti tutti questi dati, il \textit{Simulation Management Module} esegue una simulazione dell'automazione che l'utente aveva intenzione di implementare nel sistema. 
Tale simulazione non solo verifica il corretto funzionamento dell'automazione stessa, ma anche l'interazione con eventuali automazioni preesistenti. Un obiettivo fondamentale del modulo è quello di individuare possibili conflitti che potrebbero sorgere, ad esempio automazioni che si contraddicono o che, se attivate contemporaneamente, potrebbero causare un uso inefficiente delle risorse energetiche o problemi di comfort per gli utenti.

Al termine della simulazione, il modulo produce una serie di suggerimenti e spiegazioni dettagliate destinate all'utente, evidenziando eventuali migliorie possibili o segnalando i conflitti riscontrati durante l'analisi. Queste permettono agli utenti di prendere decisioni informate riguardo all'adozione, alla modifica o alla rimozione delle automazioni proposte, contribuendo così a migliorare continuamente la gestione energetica e funzionale della propria abitazione (fig. \ref{fig:simManMod_to_user}). \\[0.4cm]

\begin{figure}[H]
    \centering
    \includegraphics[width=0.3\linewidth]{images/Chap - 3//Simulation/simulation 4.png}
    \caption{Generazione e invio di suggerimenti e spiegazioni da parte del Simulation Management Module all'utente}
    \label{fig:simManMod_to_user}
\end{figure}

Nel caso invece in cui la simulazione confermi l'assenza di problemi, il \textit{Simulation Management Module} inoltra automaticamente le automazioni approvate al \textit{Home Assistant Integration Module}, che provvede poi all'effettiva implementazione e attivazione delle automazioni all'interno della smart home (fig. \ref{fig:home_to_simulation}). In questo modo, il modulo assicura un passaggio fluido ed efficiente dalla simulazione teorica alla reale implementazione delle automazioni, minimizzando il rischio di imprevisti operativi o inefficienze. 

I comandi forniti dall'utente vengono inizialmente acquisiti dalla \textit{Digital Twin Interface}, che provvede poi a inoltrarli al \textit{Simulation Management Module} (fig. \ref{fig:dig_twin_to_simul}). Quest'ultimo li utilizzerà per eseguire i necessari controlli e simulazioni, come descritto precedentemente per l'implementazione di nuove automazioni. \\[0.4cm]

\begin{figure}[H]
\centering
\includegraphics[width=.8\linewidth]{images/Chap - 3//Simulation/simulation 8.png}
\caption{Ricezione da parte dell'interfaccia del Digital Twin dei comandi dati dall'utente}
\label{fig:dig_twin_to_simul}
\end{figure}

Successivamente, al termine delle simulazioni, il \textit{Simulation Management Module} inoltra i comandi validati al \textit{Home Assistant Integration Module} (fig. \ref{fig:sim_to_home}), il quale li applica concretamente nella Smart Home fisica. \\[0.4cm]

\begin{figure}[H]
\centering
\includegraphics[width=0.3\linewidth]{images/Chap - 3//Simulation/simulation 5.png}
\caption{Invio dei comandi utente da parte del Simulation Management Module al Home Assistant Integration Module}
\label{fig:sim_to_home}
\end{figure}

Nel mentre al \textit{Digital Twin Interface} vengono inviati tutti i suggerimenti e spiegazioni generati durante la fase di simulazione (fig. \ref{fig:sim_to_dig}).
Questo permetterà quindi di andare ad integrarli all'interno della GUI che l'utente utilizza. \\[0.4cm]

\begin{figure}[H]
    \centering
    \includegraphics[width=.8\linewidth]{images/Chap - 3//Simulation/simulation 9.png}
    \caption{Invio dei suggerimenti generati durante la fase di simulazione all'interfaccia del Digital Twin}
    \label{fig:sim_to_dig}
\end{figure}

In sintesi, il \textit{Simulation Management Module} rappresenta un elemento fondamentale per assicurare che le automazioni definite dagli utenti siano sempre ottimali, prive di conflitti e perfettamente allineate agli obiettivi di sostenibilità, efficienza e comfort tipici di una smart home evoluta.

\section{Home Assistant Integration Module}

Il \textit{Home Assistant Integration Module}, rappresenta una componente centrale dell'architettura del Gemello Digitale della Smart Home. Questo modulo svolge una serie di funzioni essenziali che garantiscono la comunicazione tra il sistema di gestione smart home, rappresentato da \textit{Home Assistant}, e il resto dell'architettura del Gemello Digitale.

Innanzitutto, il modulo ha il compito di raccogliere in maniera continua i dati provenienti dai dispositivi smart integrati nell'ambiente domestico tramite \textit{Home Assistant}. 
Questi dati, oltre ad includere gli stati attuali di ogni dispositivo, forniscono anche informazioni inerenti a come i dispositivi sono stati impostati dall'utente (o da alcune automazioni), ad esempio come la temperatura impostata mediante un termostato o la luminosità di lampade.

Oltre ad acquisire le informazioni descritte, l'\textit{Home Assistant Integration Module} recupera anche i dati di tutte le automazioni configurate in \textit{Home Assistant}. Questa operazione assicura che il sistema disponga di un quadro completo e aggiornato delle automazioni attive.

La sezione di architettura che si occupa di queste funzionalità è mostrata in (fig. \ref{fig:ch3_ha_to_hoim}).

\begin{figure}[H]
    \centering
    \includegraphics[width=.7\linewidth]{images/Chap - 3//Home Integration/home_1.png}
    \caption{Ricezione dello stato dei dispositivi, del loro storico e delle automazioni}
    \label{fig:ch3_ha_to_hoim}
\end{figure}

Parallelamente allo stato corrente, il modulo acquisisce anche uno storico dettagliato dei consumi energetici generati dai vari dispositivi, come elettrodomestici, luci intelligenti, climatizzatori e sistemi di riscaldamento. Questo registro permette non solo di avere una panoramica chiara delle abitudini di utilizzo degli utenti, ma è anche fondamentale per analisi future e per la realizzazione di previsioni accurate riguardo al consumo energetico domestico.

Una volta acquisiti, questi dati di consumo vengono inoltrati a un database specificamente predisposto (fig. \ref{fig:ha_to_db}). Tale database non rappresenta semplicemente un archivio passivo, ma costituisce una componente che rende disponibili le informazioni raccolte per successive analisi predittive. Grazie a queste informazioni dettagliate e cronologicamente ordinate, il sistema può identificare pattern, rilevare anomalie nei consumi e suggerire strategie di ottimizzazione energetica agli utenti. \\[0.4cm]

\begin{figure}[H]
    \centering
    \includegraphics[width=.8\linewidth]{images/Chap - 3//Home Integration/home_2.png}
    \caption{Invio dei dati di consumo al database}
    \label{fig:ha_to_db}
\end{figure}

Oltre alla raccolta e archiviazione dei dati di consumo, il \textit{Home Assistant Integration Module} mantiene costantemente aggiornata una lista di tutte le automazioni configurate in \textit{Home Assistant}. Questa lista dettagliata e aggiornata è cruciale per garantire che la \textit{Digital Twin Interface} disponga sempre delle informazioni più recenti circa le automazioni attive (fig. \ref{fig:ha_to_dti}). Ciò consente agli utenti del Gemello Digitale di visualizzare chiaramente tutte le automazioni esistenti, gestirle e modificarle in tempo reale. \\[0.4cm]

\begin{figure}[H]
    \centering
    \includegraphics[width=.8\linewidth]{images/Chap - 3//Home Integration/home_3.png}
    \caption{Invio delle automazioni alla Digital Twin Interface}
    \label{fig:ha_to_dti}
\end{figure}

Infine il modulo funge anche da intermediario per la comunicazione verso il sistema \textit{Home Assistant}. Nello specifico, riceve dal \textit{Simulation Management Module} le automazioni e i comandi manuali generati dagli utenti in fase di simulazione e verifica. Questi comandi, dopo essere stati validati attraverso le simulazioni, vengono trasmessi a \textit{Home Assistant} per la loro implementazione effettiva nell'ambiente domestico reale (fig. \ref{fig:haim_to_ha}). In questo modo, il modulo assicura che l'applicazione delle automazioni simulate sia accurata e coerente con le intenzioni degli utenti, riducendo il rischio di errori o conflitti operativi. \\[0.4cm]

\begin{figure}[H]
    \centering
    \includegraphics[width=.6\linewidth]{images/Chap - 3//Home Integration/home_4.png}
    \caption{Invio dei comandi e delle automazioni a Home Assistant}
    \label{fig:haim_to_ha}
\end{figure}

Attraverso queste funzionalità integrate e coordinate, il \textit{Home Assistant Integration Module} gioca un ruolo fondamentale nel mantenere l'intera architettura del Gemello Digitale sincronizzata, aggiornata e perfettamente funzionante, garantendo un'elevata affidabilità e una user experience ottimale.

\section{Configuration Management Module}
Il modulo di gestione della configurazione svolge la funzione di raccogliere e conservare in modo coerente tutte le preferenze espresse dagli utenti, gli obiettivi definiti e le specifiche impostazioni di configurazione che questi hanno precedentemente inserito nel sistema. Tali informazioni vengono memorizzate all'interno del database, garantendo così la disponibilità continua e l'aggiornamento tempestivo dei dati rilevanti. 

Per poter modificare tali dati è possibile accedervi tramite l'interfaccia del Digital Twin (fig. \ref{fig:config_1}).\\[0.4cm]

\begin{figure}[H]
    \centering
    \includegraphics[width=.325\linewidth]{images/Chap - 3/Configuration/configuration_1.png}
    \caption{Porzione di architettura che si occupa dell'accesso e modifica delle configurazioni dell'utente}
    \label{fig:config_1}
\end{figure}


\section{Data Analysis Module}

Il \textit{Data Analysis Module} rappresenta una componente fondamentale all'interno dell'architettura presa in esame, svolgendo il ruolo cruciale di raccogliere, elaborare e interpretare i dati energetici generati dagli utenti e dai dispositivi domestici. Attraverso processi di analisi avanzati, questo modulo permette di ottenere informazioni approfondite e utili per ottimizzare la gestione energetica della casa, migliorando l'efficienza e la sostenibilità complessiva del sistema.

In particolare, il modulo svolge tre funzioni primarie:

\begin{enumerate}
\item \textbf{Acquisizione e validazione dei dati}: estrae dal database tutti i dati storici di consumo energetico provenienti dai dispositivi integrati in \textit{Home Assistant}.
\item \textbf{Elaborazione e modellazione}: i dati vengono arricchiti con variabili contestuali (ad esempio il giorno della settimana e la fascia oraria) e sottoposti a pipeline di \textit{feature engineering} per alimentare diversi modelli di Machine Learning (ML) e Deep Learning (DL) impiegati per analisi descrittive, predittive e prescrittive.
\item \textbf{Servizi di esposizione}: rende disponibili, tramite API REST, le previsioni fatte alla \textit{Digital Twin Interface} e al \textit{Simulation Management Module}, consentendo così la verifica della sostenibilità e il supporto alle decisioni operative.
\end{enumerate}

Tutti questi processi vengono eseguiti dalla seguente porzione di architettura (fig. \ref{fig:dam}).\\[0.4cm]

\begin{figure}[H]
\centering
\includegraphics[width=0.3\linewidth]{images/Chap - 3//Data/data_1.png}
\caption{Porzione di architettura che si occupa di effettuare le predizioni sui consumi}
\label{fig:dam}
\end{figure}

In particolare, per la previsione dei consumi futuri è stato adottato un approccio basato su modelli di \textit{Long Short-Term Memory} (LSTM), che risultano particolarmente efficaci nell'analisi di dati temporali grazie alla loro capacità di apprendere dipendenze sequenziali anche su intervalli temporali lunghi \cite{guizzardi2025user}. I modelli sono stati allenati su dati storici di consumo energetico provenienti dall’ambiente domestico reale, arricchiti con informazioni contestuali come il giorno della settimana e la fascia oraria. Le predizioni generate vengono poi impiegate per valutare, in fase preventiva, l’impatto delle automazioni sulla sostenibilità energetica dell’abitazione, supportando l’utente nel prendere decisioni più consapevoli e informate. Per la selezione dei modelli più performanti è stata condotta una procedura di ottimizzazione iperparametrica mediante \textit{grid search}, in grado di individuare le combinazioni ottimali dei parametri principali del modello in base all’accuratezza predittiva. Tutti i dettagli di questa metodologia sono descritti in \cite{guizzardi2025user}.

\newpage
\section{Smart Home e Home Assistant}
La \textit{Smart Home} rappresenta l'ambiente fisico reale in cui vengono applicate e sperimentate le automazioni create e gestite dal sistema. Questo ambiente comprende una varietà di dispositivi domestici intelligenti, tra cui elettrodomestici, sistemi di illuminazione, climatizzazione, sensori ambientali e dispositivi di sicurezza. Questi dispositivi comunicano costantemente con il sistema centrale tramite \textit{Home Assistant}, che funge da ponte essenziale tra gli utenti, il Gemello Digitale e l'ambiente fisico.

\textit{Home Assistant} svolge principalmente due ruoli fondamentali. In primo luogo, raccoglie continuamente i dati provenienti dai dispositivi domestici, registrandone lo stato corrente (ad esempio acceso, spento, temperatura rilevata, consumo energetico) e notificando eventuali cambiamenti di stato in tempo reale al sistema centrale. In secondo luogo, \textit{Home Assistant} riceve i comandi generati dall'utente attraverso l'interfaccia digitale o elaborati dal Gemello Digitale e li inoltra direttamente ai dispositivi appropriati. Ciò assicura che le automazioni desiderate dagli utenti siano correttamente implementate, garantendo così un'efficiente interazione e un elevato grado di integrazione tra il mondo fisico reale e la sua controparte digitale.

Questa gestione bidirezionale dei dati e dei comandi consente di mantenere sincronizzati e aggiornati costantemente entrambi gli ambienti, permettendo al Gemello Digitale di simulare fedelmente e prevedere con precisione le condizioni operative della Smart Home, migliorando così la qualità e l'efficienza complessiva della gestione domestica (fig. \ref{fig:smart_home_fisica}).\\[0.4cm]

\begin{figure}[H]
\centering
\includegraphics[width=0.35\linewidth]{images/Chap - 3//Home/home_fis_1.png}
\caption{Smart Home fisica}
\label{fig:smart_home_fisica}
\end{figure}

\section{Definizione delle automazioni da parte dell'utente}

Sebbene l'architettura descritta copra in maniera approfondita aspetti cruciali quali la simulazione preventiva delle automazioni, l'integrazione con \textit{Home Assistant} e la gestione avanzata dei dati energetici, rimane ancora incompleta la parte riguardante la definizione delle automazioni da parte dell'utente finale.

Al momento, l'interfaccia grafica presentata consiste in un prototipo che consente principalmente l'attivazione e la disattivazione intuitiva di automazioni predefinite tramite uno slider, nonché la visualizzazione di suggerimenti energetici generati a seguito delle simulazioni preventive. Tuttavia, essa non permette ancora agli utenti di creare da zero nuove automazioni personalizzate, definendo in autonomia le condizioni specifiche di attivazione (trigger), le azioni conseguenti e i dispositivi coinvolti.

La progettazione e implementazione completa di questa componente, che costituirà un elemento fondamentale del sistema, sarà trattata nel prossimo capitolo. Tale capitolo approfondirà in dettaglio come gli utenti potranno interagire con una GUI dedicata, semplice e intuitiva, per definire autonomamente le proprie automazioni.

Questo sviluppo contribuirà significativamente a estendere le capacità del sistema, valorizzando ulteriormente l'interazione utente-sistema e promuovendo una gestione domestica ancora più flessibile.

    \chapter{Progettazione e implementazione dell'interfaccia utente}
\label{chap:design-implementation}

\section{Introduzione al capitolo}
Questo capitolo approfondisce la progettazione dell'interfaccia utente del sistema sviluppato, con l'obiettivo di mostrare come l'integrazione delle funzionalit\`a di Trigger-Action Programming (TAP) nel Gemello Digitale sia stata tradotta in un'esperienza di interazione coerente, controllabile e aderente ai requisiti discussi nei capitoli precedenti. L'interfaccia non \`e un semplice elemento di presentazione, ma il punto di accesso privilegiato per osservare lo stato della Smart Home e per definire automazioni che, dopo la fase di simulazione, vengono applicate alla componente fisica.

In continuit\`a con l'architettura descritta nel Capitolo~\ref{chap:digital-twin-smart-home}, l'interfaccia svolge una funzione di mediazione tra l'utente e i moduli del sistema: rende percepibili i dati generati dal Gemello Digitale, permette di interpretare i risultati della simulazione e guida l'utente nella creazione e gestione delle regole TAP. Il capitolo si colloca quindi come ponte tra la visione concettuale e l'implementazione, motivando le scelte di design e illustrando il percorso dal prototipo all'interfaccia funzionante.

\section{Analisi dei requisiti dell'interfaccia}
L'analisi dei requisiti prende avvio dal contesto applicativo della Smart Home e dal paradigma TAP presentato nel Capitolo 2, nonch\'e dalla struttura modulare del Gemello Digitale descritta nel Capitolo~\ref{chap:digital-twin-smart-home}. In questo quadro, l'interfaccia deve rendere accessibile un insieme eterogeneo di funzionalit\`a, senza introdurre complessit\`a non necessarie o ambiguit\`a nella gestione delle automazioni.

Dal punto di vista funzionale, l'interfaccia deve consentire la consultazione dello stato del sistema, la visualizzazione dei consumi energetici e delle previsioni, la creazione e modifica di regole trigger-action, la gestione delle configurazioni dell'abitazione e delle preferenze utente, nonch\'e l'esecuzione di controlli manuali quando necessario. \`E inoltre essenziale fornire un accesso esplicito agli esiti delle simulazioni, includendo eventuali conflitti o suggerimenti, poich\'e la validazione preventiva costituisce un requisito cardine del sistema.

I requisiti di usabilit\`a e chiarezza discendono dalla natura del dominio: le automazioni influenzano il comportamento della casa reale e un errore di configurazione pu\`o produrre conseguenze tangibili. Per questo motivo l'interfaccia deve privilegiare la visibilit\`a dello stato, il feedback tempestivo sulle azioni, la tracciabilit\`a delle informazioni e l'adozione di una terminologia coerente con quella introdotta nei capitoli precedenti. La prevenzione degli errori non si limita a messaggi di avviso, ma richiede una progettazione che renda evidenti le relazioni tra trigger, condizioni e azioni, riducendo l'ambiguit\`a nelle scelte dell'utente.

La tipologia di utenti prevista comprende soggetti con livelli di competenza differenti: utenti domestici interessati a monitorare lo stato della casa e a configurare automazioni semplici, utenti tecnici chiamati a definire mappe e dispositivi, e profili orientati all'analisi dei consumi che richiedono viste sintetiche ma confrontabili nel tempo. La progettazione ha quindi dovuto conciliare immediatezza e profondit\`a informativa, prevedendo percorsi che consentano sia un utilizzo rapido, sia un accesso graduato ai dettagli.

\section{Progettazione dell'interfaccia utente}
\subsection{Approccio progettuale e criteri guida}
La progettazione ha seguito un approccio user-centered, con iterazioni mirate alla definizione dei principali flussi di interazione: monitoraggio dello stato della casa, consultazione dei consumi, definizione delle regole TAP e gestione della configurazione. La coerenza con il Gemello Digitale ha richiesto che ogni entit\`a significativa nel sistema, come ambienti, dispositivi e automazioni, fosse riconoscibile nell'interfaccia con una terminologia stabile e una rappresentazione uniforme.

Tra i criteri guida si \`e privilegiata la semplicit\`a operativa, intesa come riduzione del carico cognitivo e come minimizzazione dei passi necessari per completare un compito. Parallelamente, la modularit\`a dell'interfaccia \`e stata considerata essenziale per garantire l'estendibilit\`a del sistema e per permettere una separazione netta tra consultazione e azione. Tale separazione, oltre a favorire l'usabilit\`a, risponde alla necessit\`a di prevenire attivazioni non intenzionali in un contesto in cui le automazioni incidono sul mondo fisico.

\subsection{Layout, navigazione e organizzazione delle informazioni}
L'architettura informativa \`e stata organizzata in sezioni principali, accessibili tramite una navigazione persistente. La dashboard assolve il ruolo di sintesi dello stato del Gemello Digitale, mentre le sezioni dedicate a consumi, automazioni e configurazione consentono di approfondire aspetti specifici senza perdere il contesto generale. Questa scelta risponde alla necessit\`a di mantenere il controllo percettivo del sistema anche quando si accede a funzionalit\`a avanzate.

L'organizzazione interna delle schermate segue una logica di progressiva rivelazione: gli indicatori essenziali sono presentati in modo immediato, mentre i dettagli vengono esposti solo quando l'utente decide di approfondire. Tale scelta \`e motivata dall'eterogeneit\`a dei dati trattati dal Gemello Digitale e dalla necessit\`a di evitare un sovraccarico informativo, specialmente per gli utenti meno esperti. La disposizione dei controlli di azione in aree dedicate, distinte dalle aree di sola consultazione, rafforza inoltre la chiarezza operativa e la prevenzione degli errori.

\subsection{Rappresentazione delle regole Trigger-Action e feedback}
L'integrazione del paradigma TAP ha richiesto una rappresentazione che rendesse esplicita la sequenza logica di trigger, condizioni e azioni. L'interfaccia \`e stata quindi progettata per guidare l'utente attraverso la composizione delle regole, assicurando che ogni passaggio sia comprensibile e verificabile. La scelta di una struttura modulare risponde alla necessit\`a di mantenere la flessibilit\`a del paradigma TAP senza introdurre ambiguit\`a nei legami tra gli elementi della regola.

Un aspetto centrale riguarda la restituzione dei risultati della simulazione. Poich\'e la validazione preventiva \`e parte integrante dell'architettura, l'interfaccia deve comunicare in modo chiaro gli esiti, includendo eventuali conflitti e suggerimenti. Ci\`o richiede un sistema di feedback che distingua tra avvisi informativi e blocchi operativi, conservando la trasparenza dell'intero processo. In tal modo l'utente non percepisce la simulazione come un vincolo opaco, ma come un supporto decisionale coerente con gli obiettivi del Gemello Digitale.

\section{Prototipazione tramite FIGMA}
La prototipazione ad alta fedelt\`a \`e stata realizzata in FIGMA per validare la struttura informativa e il flusso di interazione prima dell'implementazione. Le immagini del prototipo sono disponibili nella cartella \texttt{Tesi\_Magistrale\_Golino\_Giacomo\textbackslash images\textbackslash Chap - 4} e vengono richiamate nelle figure di questa sezione per illustrare le principali schermate.

\subsection{Schermate principali}
La sezione dedicata ai consumi energetici, mostrata in Figura~\ref{fig:cap4_consumptions}, rappresenta il punto di accesso alle analisi del consumo e alle previsioni. La schermata \`e concepita per supportare il confronto tra intervalli temporali differenti e per distinguere tra consumo totale, contributo delle automazioni e stime predittive. Nel prototipo sono presenti varianti della vista che mostrano tali configurazioni, coerenti con i requisiti di supporto alle decisioni energetiche.

\begin{figure}[H]
    \centering
    \includegraphics[width=1\linewidth]{images/Chap - 4/Consumptions.png}
    \caption{Prototipo della sezione di analisi dei consumi energetici.}
    \label{fig:cap4_consumptions}
\end{figure}

La gestione delle automazioni (Figura~\ref{fig:cap4_automations}) esprime la componente centrale dell'integrazione TAP nel Gemello Digitale. La schermata \`e pensata per rendere visibili le regole attive, lo stato di ciascuna automazione e l'accesso ai dettagli necessari per la modifica. In questo contesto, la relazione tra trigger, condizioni e azioni \`e mantenuta esplicita per facilitare il controllo e la comprensione delle regole.

\begin{figure}[H]
    \centering
    \includegraphics[width=1\linewidth]{images/Chap - 4/Automations.png}
    \caption{Prototipo della sezione di gestione delle automazioni trigger-action.}
    \label{fig:cap4_automations}
\end{figure}

La configurazione dell'abitazione, illustrata in Figura~\ref{fig:cap4_house_configuration}, supporta la mappatura tra spazi fisici e rappresentazione digitale. Tale passaggio \`e essenziale per garantire che le automazioni TAP siano contestualizzate in modo coerente con la Smart Home reale, preservando la fedelt\`a del Gemello Digitale.

\begin{figure}[H]
    \centering
    \includegraphics[width=0.95\linewidth]{images/Chap - 4/House configration_final_1.png}
    \caption{Prototipo della configurazione dell'abitazione e dei dispositivi.}
    \label{fig:cap4_house_configuration}
\end{figure}

L'area profilo (Figura~\ref{fig:cap4_profile}) integra la gestione delle informazioni personali e delle preferenze dell'utente, consentendo di collegare le impostazioni individuali ai parametri utilizzati dai moduli di simulazione e analisi. Questa schermata \`e fondamentale per mantenere coerenza tra obiettivi dell'utente e comportamento del sistema.

\begin{figure}[H]
    \centering
    \includegraphics[width=0.7\linewidth]{images/Chap - 4/Profile.png}
    \caption{Prototipo dell'area profilo e delle preferenze utente.}
    \label{fig:cap4_profile}
\end{figure}

La vista sull'impronta ecologica (Figura~\ref{fig:cap4_ecological}) fornisce una sintesi orientata alla sostenibilit\`a, utile per collegare il comportamento delle automazioni agli obiettivi energetici del sistema. La presenza di questa schermata nel prototipo rafforza la connessione tra TAP e ottimizzazione dei consumi.

\begin{figure}[H]
    \centering
    \includegraphics[width=0.9\linewidth]{images/Chap - 4/Ecological footprint.png}
    \caption{Prototipo della vista di sintesi dell'impronta ecologica.}
    \label{fig:cap4_ecological}
\end{figure}

\subsection{Elementi di supporto e coerenza visiva}
Oltre alle schermate principali, il prototipo include elementi di supporto che contribuiscono alla coerenza visiva e al feedback operativo. La definizione della palette cromatica (Figura~\ref{fig:cap4_palette}) esplicita i ruoli dei colori per stati, avvisi e informazioni neutre, garantendo una leggibilit\`a uniforme tra le sezioni. L'uso di tali convenzioni \`e funzionale alla chiarezza e alla riconoscibilit\`a delle informazioni.

\begin{figure}[H]
    \centering
    \includegraphics[width=0.7\linewidth]{images/Chap - 4/Color palette.png}
    \caption{Palette cromatica utilizzata nel prototipo.}
    \label{fig:cap4_palette}
\end{figure}

Il prototipo comprende inoltre esempi di notifiche contestuali per comunicare l'esito di azioni o variazioni di stato, come mostrato in Figura~\ref{fig:cap4_feedback}. Questi elementi sono rilevanti per la percezione di affidabilit\`a del sistema, in quanto informano l'utente sugli effetti delle operazioni eseguite.

\begin{figure}[H]
    \centering
    \includegraphics[width=0.7\linewidth]{images/Chap - 4/Achievements popup.png}
    \caption{Esempio di notifica contestuale nel prototipo.}
    \label{fig:cap4_feedback}
\end{figure}

\section{Scelte implementative}
La traduzione del prototipo in un'interfaccia funzionante ha richiesto decisioni che preservassero la coerenza concettuale e grafica, garantendo al contempo affidabilit\`a e manutenibilit\`a. L'adozione di un'architettura a componenti permette di riflettere la modularit\`a del design, assicurando uniformit\`a tra le sezioni e facilitando l'evoluzione futura dell'interfaccia. Tale impostazione supporta la riusabilit\`a di elementi comuni e riduce la probabilit\`a di inconsistenze visive o comportamentali.

L'interfaccia deve gestire dati dinamici provenienti da moduli diversi, con aggiornamenti asincroni e potenziali ritardi. Per evitare discrepanze percettive tra stato reale e rappresentazione digitale, sono stati previsti meccanismi di aggiornamento chiari e indicatori che rendono esplicito il momento di sincronizzazione. Questo aspetto \`e particolarmente rilevante per le automazioni TAP, poich\'e l'utente deve distinguere tra una regola definita e una regola effettivamente validata dalla simulazione.

Un ulteriore elemento riguarda la validazione delle regole e la prevenzione degli errori. Il flusso di definizione delle automazioni \`e stato implementato in modo guidato, con controlli che rendono evidenti i passaggi necessari e con feedback che segnalano incompletezze o conflitti prima dell'invio al modulo di simulazione. Tale scelta privilegia la sicurezza operativa rispetto alla mera rapidit\`a di configurazione, in linea con la natura critica delle automazioni in un ambiente domestico reale.

Nel passaggio dal prototipo all'implementazione sono stati affrontati compromessi legati alla densit\`a informativa. Alcuni dettagli avanzati sono stati mantenuti in sezioni dedicate, cos\`i da preservare la leggibilit\`a delle viste principali. Inoltre, la coesistenza con strumenti conversazionali esterni per la creazione delle automazioni, discussa nel Capitolo~\ref{chap:digital-twin-smart-home}, ha richiesto di mantenere un percorso coerente tra le diverse modalit\`a di interazione, evitando duplicazioni non necessarie.

\section{Considerazioni finali}
La progettazione dell'interfaccia utente ha seguito un percorso coerente con i requisiti e con l'architettura del Gemello Digitale, integrando il paradigma TAP in un'interazione comprensibile e verificabile. Le scelte effettuate hanno privilegiato la trasparenza del processo di simulazione, la chiarezza terminologica e la prevenzione degli errori, bilanciando la complessit\`a del dominio con l'esigenza di usabilit\`a.

Il capitolo conclusivo riprende queste scelte per discutere il sistema nel suo complesso e per sintetizzare le implicazioni progettuali e applicative emerse, fornendo un quadro finale delle opportunit\`a e delle prospettive di evoluzione.

    \backmatter
    \chapter{Conclusioni}
\lipsum[5]
    \printbibliography
\end{document}
